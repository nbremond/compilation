\documentclass[a4paper,10pt]{report}

% \usepackage{ gensymb }
\usepackage[utf8]{inputenc}
\usepackage[T1]{fontenc}
\usepackage[french]{babel}
\usepackage[vlined, lined, linesnumbered, boxruled, french]{algorithm2e}
\usepackage{xcolor,amsmath,amssymb}
\usepackage[babel=true]{csquotes}
\usepackage{fancyhdr}
\usepackage{graphicx}
\pagestyle{fancy}
\usepackage{float}
\usepackage{setspace}
\usepackage{listings}
\usepackage{fancyvrb}
%%%%%%%%%%%%%%%% Lengths %%%%%%%%%%%%%%%%
\setlength{\textwidth}{15.5cm}
\setlength{\evensidemargin}{0.5cm}
\setlength{\oddsidemargin}{0.5cm}

\setlength{\voffset}{-.7in}
\setlength{\hoffset}{-.1in}
\setlength{\textheight}{24.7cm}

\renewcommand*\thesection{\arabic{section}} 
%%%%%%%%%%%%%%%% Variables %%%%%%%%%%%%%%%%
% \def\projet{7}
\def\titre{Projet de Compilation}
% \def\groupe{3}
\def\others{Nicolas Bremond, Lotfi Zouad, Valentin Darricau, Nicolas Richard}
\newtheorem{preuve}{\textit{Preuve }}
\begin{document}
\fancyhfoffset{0cm}{}
%%%%%%%%%%%%%%%% Header and footer %%%%%%%%%%%%%%%%
\noindent\begin{minipage}{0.98\textwidth}
  \vskip 0mm
  \noindent
  { \begin{tabular}{p{4.5cm}}
      \centering \bfseries
      {\itshape \titre} \\
      % {\sffamily Projet n\degree \projet}         
    \end{tabular}}
  \hfill 
  \fbox{\begin{tabular}{l}
      {~\hfill \bfseries \sffamily Groupe %\groupe\ %- \'Equipe \equipe
        \hfill~} \\[2mm] 
      % Coordinateur : \responsible \\
      Codeurs : \others
    \end{tabular}}
  \vskip 4mm ~ 

  
  \vskip 1mm
\end{minipage}

\lhead{ \titre}
\chead{}
\rhead{\others}

\lfoot{ENSEIRB-MATMECA}
\cfoot{}
\rfoot{Semestre 7 2014-2015}

%%%%%%%%%%%%%%%% Main part %%%%%%%%%%%%%%%%

\section{Choix de l'extension}

Nous avons choisis l'extension <<Vérification statique et dynamique du
code>>. Cette extension permettra de vérifier les dépassements de
tableaux et la présence de <<return>> en fin de programme dans les
fonctions qui ne retournent pas <<void>>.

\section{Tests}

\subsubsection{Tests pour les dépassements de tableaux}



Ces tests permettent de vérifier que le dépassement dans les
tableaux est bien géré par notre compilateur.\\
Ils vérifient avec la déclaration statique et la déclaration avec les mallocs.\\

\begin{itemize}
\item Test 1_1:
  \begin{Verbatim}
    float a[100];
    a[-1] = 0; 
  \end{Verbatim}

\item Test 1_2:
  \begin{Verbatim}
    float * a = malloc(sizeof(float)*100);
    a[-1] = 0; 
  \end{Verbatim}

\item Test1_3:
  \begin{Verbatim}
    float a[100];
    int i;
    for (i=0; i<=100; i++) {
      a[i] = 0;
    }
  \end{Verbatim}

\item Test 1_4:
  \begin{Verbatim}
    float * a = malloc(sizeof(float)*100);
    int i;
    for (i=0; i<=100; i++) {
      a[i] = 0;
    }
  \end{Verbatim}

  Ces quatre derniers tests vérifient également les dépassements de tableau mais d'une manière différente. Ce dépassement de tableau dépend de plus d'un paramètre.
\item Test 1_5:
  \begin{Verbatim}
    float a[100];
    int i;
    for (i=0; i<=100; i++) {
      a[i+j] = 0; \textit{//ici il n'y a pas seulement i à surveiller}
    }
  \end{Verbatim}

\item Test 1_6:
  \begin{Verbatim}
    float a[99];
    int i;
    for (i=0; i<=33; i++) {
      a[3*i] = 0;
    }
  \end{Verbatim}


\item Test 1_7:
  \begin{Verbatim}
    float *a = malloc(sizeof(float)*99);
    int i;
    for (i=0; i<=33; i++) {
      a[3*i] = 0;
    }

  \end{Verbatim}

  Pour cette dernière fonction, il peut y avoir dépassement de tableau ou pas. Ce test ne donnera pas d'erreur de compilation.
\item Test 1_8:
  \begin{Verbatim}
    void fonction(int n){
      int i = 0;
      float a[100];
      while(i<n){
        a[i] = 0.2;
        i++;
      }
    }
  \end{Verbatim}
  \subsubsection{Tests pour les retours de fonction}

\item Test 2_1:
  \begin{Verbatim}
    int fonction(){
      int a = 5;
    }
  \end{Verbatim}

\item Test 2_2:
  \begin{Verbatim}
    int fonction(){
      float a = 5.2;
      return a;
    }
  \end{Verbatim}

\item Test 2_3:
  \begin{Verbatim}
    float fonction(){
      float a = 5.2;
      if(a<0){
        return a;
      }
      else if(a>10){
        return (a+1);
      }
    }
  \end{Verbatim}

\item Test 2_4:
  \begin{Verbatim}
    float fonction(int i){
      int n = 100;
      while (i<n){
        if (i == 80){
          return i;
        }      
        i++;
      }
    }
  \end{Verbatim}



\end{itemize}
\newpage

\end{document}

